\PassOptionsToPackage{table}{xcolor}
%\documentclass{beamer}
\documentclass[compress]{beamer}
%\usetheme{Epam}
\usetheme{Copenhagen}
\usepackage{xcolor}
\usepackage{listings}
\usepackage{graphicx}
\usepackage[utf8]{inputenc}
\usepackage{datetime}
\usepackage{beamerthemesplit}
%\beamertemplatenavigationsymbolsempty
\usepackage{listingsutf8}
\usepackage{ragged2e}
\usepackage{hyperref}
\lstset{ %
  language=C,                      % the language of the code
%  basicstyle=\ttfamily,           % the size of the fonts that are used for the code
%  basicstyle=\ttfamily\tiny,      % the size of the fonts that are used for the code
  basicstyle=\ttfamily\scriptsize, % the size of the fonts that are used for the code
  numbers=left,                    % where to put the line-numbers
  numberstyle=\tiny\color{gray},   % the style that is used for the line-numbers
  stepnumber=1,                    % the step between two line-numbers. If it's 1, each line 
                                   % will be numbered
  numbersep=5pt,                   % how far the line-numbers are from the code
  %backgroundcolor=\color{gray},   % choose the background color. You must add \usepackage{color}
  showspaces=false,               % show spaces adding particular underscores
  showstringspaces=false,         % underline spaces within strings
  showtabs=false,                 % show tabs within strings adding particular underscores
%  frame=shadowbox,                   % adds a frame around the code
  rulecolor=\color{black},        % if not set, the frame-color may be changed on line-breaks within not-black text (e.g. commens (green here))
  tabsize=4,                      % sets default tabsize to 4 spaces
  captionpos=,                   % sets the caption-position to bottom
  breaklines=false,                % sets automatic line breaking
  breakatwhitespace=false,        % sets if automatic breaks should only happen at whitespace
  title=\lstname,                 % show the filename of files included with \lstinputlisting;
                                  % also try caption instead of title
  keywordstyle=\color{blue},      % keyword style
  commentstyle=\color{mygreen},   % comment style
  stringstyle=\color{magenta},    % string literal style
%  escapeinside={\%*}{*)},        % if you want to add a comment within your code
  inputencoding=utf8,
  extendedchars=\true,
  morekeywords={*,..., restrict, alignof, alignas, bool, true, false, size_t, ssize_t, inline, \_Noreturn, noreturn},
  breakautoindent=false,
  breakindent=1pt,
}

%\setbeameroption{show only notes}
%\usepackage{pgfpages}
%\setbeameroption{show notes}
%\setbeameroption{show notes on second screen=right}

\setbeamertemplate{navigation symbols}{}

\definecolor{oddrow}{RGB}{100,149,237}
\definecolor{evenrow}{RGB}{135,206,250}
\def\mybs{\textbackslash}
\newcommand{\qq}{\symbol{34}} % the decimal ascii code for "
\newcommand{\sq}{\symbol{39}} % the decimal ascii code for '

\makeatletter
\newcommand{\rmnum}[1]{\romannumeral #1}
\newcommand{\Rmnum}[1]{\expandafter\@slowromancap\romannumeral #1@}
\newcommand{\inc}{\symbol{45}\symbol{45}}
\newcommand{\dec}{\symbol{43}\symbol{43}}
\newcommand{\lsh}{\symbol{60}\symbol{60}}
\newcommand{\rsh}{\symbol{62}\symbol{62}}
\makeatother

\newcommand{\specialcell}[2][c]{%
  \begin{tabular}[#1]{@{}c@{}}#2\end{tabular}}
\newcommand{\specialcellhl}[2][l]{%
  \begin{tabular}[#1]{@{}l@{}}#2\end{tabular}}
\newcommand{\specialcellhc}[2][c]{%
  \begin{tabular}[#1]{@{}c@{}}#2\end{tabular}}

\definecolor{mygreen}{rgb}{0,0.6,0}
\definecolor{olive}{rgb}{0.3, 0.4, .1}
\definecolor{fore}{RGB}{249,242,215}
\definecolor{back}{RGB}{51,51,51}
\definecolor{title}{RGB}{255,0,90}
\definecolor{dgreen}{rgb}{0.,0.6,0.}
\definecolor{gold}{rgb}{1.,0.84,0.}
\definecolor{JungleGreen}{cmyk}{0.99,0,0.52,0}
\definecolor{BlueGreen}{cmyk}{0.85,0,0.33,0}
\definecolor{RawSienna}{cmyk}{0,0.72,1,0.45}
\definecolor{Magenta}{cmyk}{0,1,0,0}

\newcommand{\kwblue}[1]{\texttt{\textcolor{blue}{#1}}}
\newcommand{\kwblack}[1]{\texttt{\textcolor{black}{#1}}}
\newcommand{\kwred}[1]{\texttt{\textcolor{red}{#1}}}
\newcommand{\kwmagenta}[1]{\texttt{\textcolor{Magenta}{#1}}}

%\newcommand{\hdr}[1]{\textless{#1}\textgreater{}}
\newcommand{\hdr}[1]{\textless{\kwmagenta{#1}}\textgreater{}}

\author[\href{mailto:vasili_slapik@epam.com}{Vasili Slapik}]{\texorpdfstring{Vasili Slapik\newline\href{mailto:vasili_slapik@epam.com}{vasili\_slapik@epam.com}}{Vasili Slapik}}


\title{\Rmnum{6}. Pointers}

%TODO: strict aliasing, -fstrict-aliasing
%TODO: restrict
%TODO: summary

\begin{document}

%%%%%%%%%%%%%%%%%%%%%%%%%%%%%%%%%%%%%%%%%%%%%%%%%%%%%%%%%%%%%%%%%%%%%%%%%%%%%%%%%
\frame{\titlepage}
%%%%%%%%%%%%%%%%%%%%%%%%%%%%%%%%%%%%%%%%%%%%%%%%%%%%%%%%%%%%%%%%%%%%%%%%%%%%%%%%%
\begin{frame}{Pointer examples}
    \lstinputlisting[numbers=none]{06_example.txt}
\end{frame}
%%%%%%%%%%%%%%%%%%%%%%%%%%%%%%%%%%%%%%%%%%%%%%%%%%%%%%%%%%%%%%%%%%%%%%%%%%%%%%%%%
\begin{frame}{Pointer arithmetic}
    \only<1>{
        There are several arithmetic operations defined on pointers to data:
        \begin{itemize}
            \item adding an integer to a pointer
            \item substracting an integer from a pointer
            \item substructing two pointers from each other
            \item comparing pointers
        \end{itemize}
    }
    \only<2>{
        \lstinputlisting{06_pointer_arithm_example.c}
    }
\end{frame}
%%%%%%%%%%%%%%%%%%%%%%%%%%%%%%%%%%%%%%%%%%%%%%%%%%%%%%%%%%%%%%%%%%%%%%%%%%%%%%%%%
\begin{frame}{\kwblack{void *}}
    \only<1>{
        \lstinputlisting[numbers=none]{06_void.c}
    }
    \only<2>{
        A pointer to \kwblue{void} (\kwblue{void *}) is a general-purpose pointer used to hold references to any data type.
        \begin{itemize}
            \item you can't perform integer arithmetic on \kwblue{void *}
            \item you can't dereference \kwblue{void *}
            \item you can assign a void pointer to any data type pointer without explicit cast
            \item \kwblue{void *} have the same representation and memory alignment as a pointer to \kwblue{char}
        \end{itemize}
    }
    \only<3>{
        \justifying
        Below is the canonical example of an abstract data type (ADT) implementation for stack structure. \kwblack{stack\_t} is an \textcolor{blue}{opaque pointer}.
        \lstinputlisting{06_void_ptr_ADT.h}
    }
\end{frame}
%%%%%%%%%%%%%%%%%%%%%%%%%%%%%%%%%%%%%%%%%%%%%%%%%%%%%%%%%%%%%%%%%%%%%%%%%%%%%%%%%
\begin{frame}{Opaque pointers}
    \justifying
    \textcolor{blue}{opaque pointer} is a special case of an opaque data type, a datatype declared to be a pointer to a record or data structure of some unspecified type.

    \begin{tabular}{lll}
        \lstinputlisting{06_opaque_example_ok.c} & \: &
        \lstinputlisting{06_opaque_example_fail.c} \\
    \end{tabular}
\end{frame}
%%%%%%%%%%%%%%%%%%%%%%%%%%%%%%%%%%%%%%%%%%%%%%%%%%%%%%%%%%%%%%%%%%%%%%%%%%%%%%%%%
\begin{frame}{Null pointer}
    \only<1>{
        \justify
        For each pointer type, there is a special value \- the "null pointer" \- which is distinguishable from all other pointer values and which is "guaranteed to compare unequal to a pointer to any object or function". In other words "null pointer"\: is not the address of any object or function.
    }
    \only<2>{
        \begin{itemize}
            \item null pointer has value called null pointer constant - integer constant expression with value 0 (or (\kwblue{void} *)0)
            \item \hdr{stddef.h} usually has a definition to represent the null pointer constant such as \texttt{\kwblue{\#define} NULL ((\kwblue{void} *)0)}
            \item \justifying an initialization, assignment, or comparison when one side is a variable or expression of pointer type, the compiler can tell that a constant 0 on the other side requests a null pointer, and generate the correctly-typed null pointer value
            \item the actual internal representation of null pointers is implementation-defined; NULL and 0 are language-level symbols that represent a null pointer
            \item dereferencing of a null pointer leads to \textcolor{red}{undefined behaviour}
        \end{itemize}
    }
    \only<3>{
        \lstinputlisting{06_null_deref.c}
    }
    \only<4>{
        \lstinputlisting{06_null_pointer_mmap.c}
    }
\end{frame}
%%%%%%%%%%%%%%%%%%%%%%%%%%%%%%%%%%%%%%%%%%%%%%%%%%%%%%%%%%%%%%%%%%%%%%%%%%%%%%%%%
\begin{frame}{Wild pointers}
    \lstinputlisting{06_wild_pointer.c}
\end{frame}
%%%%%%%%%%%%%%%%%%%%%%%%%%%%%%%%%%%%%%%%%%%%%%%%%%%%%%%%%%%%%%%%%%%%%%%%%%%%%%%%%
\begin{frame}{Dangling pointers}
    \lstinputlisting{06_dangling_pointer.c}
\end{frame}
%%%%%%%%%%%%%%%%%%%%%%%%%%%%%%%%%%%%%%%%%%%%%%%%%%%%%%%%%%%%%%%%%%%%%%%%%%%%%%%%%
\begin{frame}{Pointers hygiene}
    \only<1>{
        \begin{itemize}
            \item Always initialize pointers when declare them
            \item Always check pointers before using them
            \item Always check a return value of functions which allocate memory
            \item Assign null pointer constant to pointers after deallocating memory pointed by them
        \end{itemize}
    }
    \only<2>{
            \lstinputlisting{06_good_pointers_example.h}
    }
    \only<3>{
            \lstinputlisting{06_good_pointers_example1.txt}
    }
\end{frame}
%%%%%%%%%%%%%%%%%%%%%%%%%%%%%%%%%%%%%%%%%%%%%%%%%%%%%%%%%%%%%%%%%%%%%%%%%%%%%%%%%
\begin{frame}{Function pointers}
    \only<1>{
        \begin{block}{Declaration of function}
            \lstinputlisting[numbers=none]{05_func_decl.txt}
        \end{block}
        \begin{block}{Declaration of function pointer}
            \lstinputlisting[numbers=none]{05_func_ptr_decl.txt}
        \end{block}
    }
    \only<2>{
        \lstinputlisting[basicstyle=\ttfamily\tiny]{06_func_ptr_example_sigaction.c}
    }
    \only<3>{
        \lstinputlisting{06_func_ptr_example_sigaction.h}
    }
    \only<4>{
        \lstinputlisting[basicstyle=\ttfamily\tiny]{06_func_ptr_example_qsort.c}
    }
    \only<5>{
        \lstinputlisting{06_func_ptr_example_qsort.h}
    }
\end{frame}
%%%%%%%%%%%%%%%%%%%%%%%%%%%%%%%%%%%%%%%%%%%%%%%%%%%%%%%%%%%%%%%%%%%%%%%%%%%%%%%%%
\begin{frame}{Pointers and \kwblack{const}}
    \only<1>{
        \lstinputlisting{06_const_and_ptr.c}
    }
    \only<2>{
        \lstinputlisting{06_const_and_ptr_examples.h}
    }
\end{frame}
%%%%%%%%%%%%%%%%%%%%%%%%%%%%%%%%%%%%%%%%%%%%%%%%%%%%%%%%%%%%%%%%%%%%%%%%%%%%%%%%%
\begin{frame}{Common mistakes with pointers}
    \only<1>{
        \lstinputlisting{06_p_trap1.c}
    }
    \only<2>{
        \lstinputlisting{06_p_trap2.txt}
    }
    \only<3>{
        \lstinputlisting{06_double_free.c}
    }
    \only<4>{
        \lstinputlisting{unaligned_access.c}
    }
\end{frame}
%%%%%%%%%%%%%%%%%%%%%%%%%%%%%%%%%%%%%%%%%%%%%%%%%%%%%%%%%%%%%%%%%%%%%%%%%%%%%%%%%
\end{document}
